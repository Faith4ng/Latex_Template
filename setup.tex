% !TeX root = ./document.tex
% !TeX spellcheck = none

%>>> PAGESTYLE SETTING >>>
\pagestyle{plain}
% {plain(default):页脚正中, headings:页眉正中, empty:空白}

%>>> USEPACKAGE IMPORT >>>
\usepackage[inner=2.5cm,outer=1.5cm,top=2.5cm,bottom=2.5cm]{geometry}
% 页边距设置
% 上:2.5cm; 下:2.5cm; 左:2.5cm; 右:1cm

\usepackage{ctex}
% 支持中文排版,包括字体、标点、章节标题等

\usepackage{amsmath}
% 数学公式核心宏包,提供了一些增强的数学功能,比如改进的数学环境,数学符号,数学字体等。
\usepackage{amssymb}
% 定义了 amsfonts 宏包里 msam 和 mabm 字库中全部数学符号的命令,可以提供一些 LaTeX 和 AMS 没有的数学符号
\usepackage{amsfonts}
% 提供了一些特殊的数学字体,比如欧拉字体,花体字母,黑板粗体等
\usepackage{newtxtext, newtxmath}
% 提供了 Timesnew, Palatino 风格的文本和数学字体

\usepackage{siunitx}
% 用于排版数字和单位,遵循国际单位制(SI)的规则

\usepackage{graphicx}
\usepackage{graphics}
\usepackage{subfigure}
\usepackage{float}
\usepackage{caption}
\usepackage[export]{adjustbox}
% 图片插入支持

\graphicspath{{figures/}}
% 指定默认的图形路径

\usepackage{listings}
% 代码环境宏包,提供类似IDE的显示配置
\usepackage{xcolor}
% 颜色宏包,给显示提供更多的颜色配置

%\usepackage[sort&compress]{gbt7714}

%>>> listings setting >>>
% Monokai代码风格
\lstdefinestyle{monokai}{
	backgroundcolor=\color{gray!5},
	basicstyle=\ttfamily\footnotesize,
	keywordstyle=\color{orange}\bfseries,
	identifierstyle=\color{blue},
	commentstyle=\color{gray},
	stringstyle=\color{green!60!black},
	numberstyle=\color{violet},
	showstringspaces=false,
	frame=single,
	numbers=left,
	breaklines=true,
	captionpos=t,
	tabsize=4
}