% !TeX root = ./document.tex
% !TeX spellcheck = none

%>>> PAGESTYLE SETTING >>>
\pagestyle{plain}
% {plain(default):页脚正中, headings:页眉正中, empty:空白}

%>>> USEPACKAGE IMPORT >>>
\usepackage[inner=2.5cm,outer=1.5cm,top=2.5cm,bottom=2.5cm]{geometry}
% 页边距设置
% 上:2.5cm; 下:2.5cm; 左:2.5cm; 右:1cm

% >>> 中文环境支持>>>

\usepackage{ctex}
% 支持中文排版,包括字体、标点、章节标题等

% >>> 数学环境支持

\usepackage{amsmath}
% 数学公式核心宏包,提供了一些增强的数学功能,比如改进的数学环境,数学符号,数学字体等。
\numberwithin{equation}{section}
% 公式按章节编号
\numberwithin{figure}{section}\numberwithin{table}{section}
% 图表按章节编号
\usepackage{amssymb}
% 定义了 amsfonts 宏包里 msam 和 mabm 字库中全部数学符号的命令,可以提供一些 LaTeX 和 AMS 没有的数学符号
\usepackage{amsfonts}
% 提供了一些特殊的数学字体,比如欧拉字体,花体字母,黑板粗体等
\usepackage{newtxtext, newtxmath}
% 提供了 Timesnew, Palatino 风格的文本和数学字体
\usepackage{siunitx}
% 用于排版数字和单位,遵循国际单位制(SI)的规则

% >>> 行间距调整 >>>

%\setlength{\parskip}{1.0\baselineskip}
% 调整段落间距为1.0x基本行距

\linespread{1.25}
% 调整段落内部行间距为1.25x

\usepackage{enumitem}
\setlist[enumerate]{itemsep=0.2\baselineskip,parsep=0pt,leftmargin=2\parindent}
% 调整enumerate行间距
% itemsep 是 enumitem 宏包中用于控制列表项之间垂直距离的参数。它可以接受任何有效的 LaTeX 长度值,比如 pt(磅)、mm(毫米)、cm(厘米)、in(英寸)等。例如,你可以将 itemsep 设置为 5pt 或 0.5cm。你也可以使用相对长度单位,比如 \baselineskip(基础行距)和 \parskip(段落间距)。例如,你可以将 itemsep 设置为 \baselineskip 或 0.5\parskip。

% >>> 图片插入支持 >>>

\usepackage{graphicx}
\usepackage{graphics}
\usepackage{subfigure}
\usepackage{float}
\usepackage{caption}
\usepackage[export]{adjustbox}

\renewcommand{\figurename}{Fig}
% 将图表标题中的“图”改为“Fig”
\renewcommand{\tablename}{Table}
% 将图表标题中的“图”改为“Fig”
\captionsetup[figure]{labelsep=period}
% 将图表标题中的冒号替换为标点
% {none:null, colon:冒号, period, space, quad, newline}
\captionsetup[table]{labelsep=period}
% 将图表标题中的冒号替换为标点
% {none:null, colon:冒号, period, space, quad, newline}


\graphicspath{{images/}}
% 指定默认的图形路径

% >>> 表格环境支持  >>>

\usepackage{multirow}
% 可以给表格合并单元格
\usepackage{tabularx}
% 可以自动调整列宽的表格环境
\usepackage{array}
% 设置新的列类型
\newcolumntype{C}{>{\centering\arraybackslash}X}
% 定义一个新的列类型,比如 C,来表示居中对齐的列

% >>> 代码环境支持  >>>

\usepackage{listings}
% 代码环境宏包,提供类似IDE的显示配置
\usepackage{xcolor}
% 颜色宏包,给显示提供更多的颜色配置

% listings setting
% Monokai代码风格
\lstdefinestyle{monokai}{
	backgroundcolor=\color{gray!5},
	basicstyle=\ttfamily\footnotesize,
	keywordstyle=\color{orange}\bfseries,
	identifierstyle=\color{blue},
	commentstyle=\color{gray},
	stringstyle=\color{green!60!black},
	numberstyle=\color{violet},
	showstringspaces=false,
	frame=single,
	numbers=left,
	breaklines=true,
	captionpos=t,
	tabsize=4
}

% >>> 参考文献环境设置  >>>

%\usepackage[sort&compress]{gbt7714}
% gbt-7714格式的参考文献