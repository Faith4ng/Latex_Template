% !TeX root = ./document.tex
% !TeX spellcheck = none

\begin{abstract}
	% 摘要
	% Abstract
	
	这里是一段摘要,这里是一段摘要,这里是一段摘要,这里是一段摘要,这里是一段摘要,这里是一段摘要,这里是一段摘要,这里是一段摘要,这里是一段摘要,这里是一段摘要,这里是一段摘要,这里是一段摘要,这里是一段摘要。
	
	\noindent \textbf{关键字:} 关键字1,关键字2,关键字3,……
\end{abstract}

\section*{正文结构}

% 正文
这里是正文第一段,这里是正文第一段,这里是正文第一段,这里是正文第一段,这里是正文第一段,这里是正文第一段,这里是正文第一段,这里是正文第一段,这里是正文第一段,这里是正文第一段,这里是正文第一段,这里是正文第一段,这里是正文第一段。

\section{这里是第一段}

第一段正文,第一段正文,第一段正文,第一段正文,第一段正文,第一段正文,第一段正文,第一段正文,第一段正文,第一段正文,第一段正文,第一段正文,第一段正文,第一段正文,第一段正文,第一段正文。

\subsection{这里是第一小段}

第一小段正文,第一小段正文,第一小段正文,第一小段正文,第一小段正文,第一小段正文,第一小段正文,第一小段正文,第一小段正文,第一小段正文,第一小段正文,第一小段正文,第一小段正文,第一小段正文,第一小段正文,第一小段正文。

\subsubsection{这里是第一小小段}

第一小小段正文,第一小小段正文,第一小小段正文,第一小小段正文,第一小小段正文,第一小小段正文,第一小小段正文,第一小小段正文,第一小小段正文,第一小小段正文,第一小小段正文,第一小小段正文,第一小小段正文,第一小小段正文,第一小小段正文,第一小小段正文,第一小小段正文,第一小小段正文,第一小小段正文,第一小小段正文,第一小小段正文,第一小小段正文,第一小小段正文。

\section*{图表示例}

\section{全线表}

\begin{table}[H]
	\centering
	\caption{TABLE-A}
	\begin{tabular}{|l|l|l|}
		\hline
		1 & 2 & 3 \\ \hline
		4 & 5 & 6 \\ \hline
		7 & 8 & 9 \\ \hline
	\end{tabular}
\end{table}

\section{三线表}

\begin{table}[H]
	\centering
	\caption{TABLE-B}
	\begin{tabular}{l|l|l}
		\hline
		CHIP & A   & B    \\ \hline
		AM   & 10  & 8    \\ 
		GBW  & 80M & 100M \\ \hline
	\end{tabular}
\end{table}

\section{单图}

如图(\ref{fig:EXAMPLE-A})所示

\begin{figure}[H]
	\centering
	\includegraphics[width=0.3\linewidth]{EXAMPLE-A.png}
	\caption*{heading}
	\caption{EXAMPLE-A}
	\label{fig:EXAMPLE-A}
\end{figure}

\section{多图}

如图(\ref{subfig:A})和图(\ref{subfig:B})所示

\begin{figure}[H]
	\centering
	\subfigure[name of the subfigure\label{subfig:A}]
		{\includegraphics[width=0.3\linewidth]{EXAMPLE-A.png}}
	\hspace{0.2\linewidth}
	\subfigure[name of the subfigure\label{subfig:B}]
		{\includegraphics[width=0.3\linewidth]{EXAMPLE-B.png}}
	\caption*{heading}
	\caption{name of the figure}
	\label{fig:multi-image}
\end{figure}

\section*{代码环境}

代码环境,

\begin{lstlisting}[language=C,style=monokai,caption={Hello,world}]
	#include<stdio.h>
	
	int main()
	{
		printf{"hello,world!"\n};
		reutrn 0;
	}
\end{lstlisting}

\section*{参考文献索引}

这是第一篇索引\cite{swierad2016ultra}

这是第二篇索引\cite{chen2014compact}